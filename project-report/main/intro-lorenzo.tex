\section{Intro Lorenzo}

Object detection is a fundamental and essential part of many Artificial Intelligence (AI) and Machine Learning (ML) applications. From face recognition to augmented reality or image classification, all algorithms assessing these problems must first be able to find objects in and image and then recognize what they are. In general, any application that requires an algorithm to interact with the outside world through images has to be able to recognize what is seeing in the images it is capturing.

Although many problems related to image recognition have been treated in the past (like number or hand writing recognition), the problem becomes more challenging once we try to address applications where the images fed to the algorithm are more general. In this case, the algorithm not only has to recognize what there is in the image, but also whether there is something and where it is in the image. Not only that, but the model must decide between a wider selection of objects, and images might have different illumination and color characteristics. All of these issues make the problem of image recognition a challenging topic, for which the use of Deep Learning is must.

In this report, we introduce a specific object recognition problem - Pedestrian Detection - and how we assessed a solution for it. Regarding the structure of the report, we will first give some insight in approaches previously taken, then we will talk about the different available datasets and data processing needed, to be followed by our proposed approach and results.
\subsection{Pedestrian detection}

Autonomous driving is one of the hottest topics regarding AI and ML during the last years. Dozens of different companies are working on developing both the hardware and software (such as Waymo, Baidu, Aurora or Tesla) required for it, and being able to accurately detect pedestrians is a fundamental part of this approach. Most of these endeavors are dependent on being safe for third-party pedestrians, as any accident in this regards has created a public backlash on this field. For that, any model for autonomous driving must first be completely safe for pedestrians, and its ability to detect them will be a major stepping stone for the adoption of autonomous driving.

To show the importance of this topic, we should go back to the 2005 DARPA Grand Challenge to see the first large public support for an autonomous driving competition, that fueled the advanced of the field in the following 15 years. Fast forward to present day, and we can see how large and competitive this market has become. As a token of cutthroat this industry can be, there have been numerous allegations of industrial espionage and intellectual property being stolen, and almost all big firms have had to deal with this issue at least once in their short-life story (for example, both Tesla and Google have claimed to have their intellectual property stolen at some point).

This shows how important the industry of autonomous driving is, and the crucial role that pedestrian recognition plays in developing safe and reliable software to fuel this advances even further.
